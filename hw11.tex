\documentclass{exam}

\usepackage[tmargin=0.5in,rmargin=0.75in,lmargin=0.75in,bmargin=0.5in]{geometry}
\usepackage{hyperref}
\hypersetup{
  colorlinks,
  linkcolor={black},
  citecolor={black},
  urlcolor={blue!80!black}
}
\usepackage{amsmath, amsfonts, mathtools, amsthm, amssymb}
\usepackage{pdfpages}
\usepackage{enumitem}
\usepackage{xcolor}
\usepackage{tcolorbox}
\setlist[enumerate]{label=(\alph*)}

\definecolor{mycolor}{RGB}{90, 177, 200}
\definecolor{lightcolor}{RGB}{250, 254, 255}
\newtcolorbox{mybox}[1][]{%
    colback=lightcolor,
    colframe=mycolor,
}

\usepackage{pifont}

% Define a command for rating stars
\newcommand{\Rating}[1]{%
    \hspace{-1em}%
    \foreach \n in {1,...,#1}{\ding{72}\hspace{0.1em}}%
}

% Define the rating environment
\newenvironment{rating}{%
    \par
    \vspace{\baselineskip}
    \begin{center}
    \begin{tcolorbox}[colback=lightcolor,colframe=mycolor, boxrule=0.5mm, boxsep=3mm, left=2mm, right=2mm, top=0.0mm, bottom=2mm, width=2.3in, height=.4in]
        \textbf{Proof Difficulty:} \hspace{0.5em}%
}{%
    \end{tcolorbox}
    \end{center}
    \vspace{\baselineskip}
    \par
}

\newcommand\N{\mathbb{N}}
\newcommand\R{\mathbb{R}}
\newcommand\Z{\mathbb{Z}}
\renewcommand\O{\emptyset}
\renewcommand{\emptyset}{\varnothing}
\newcommand\Q{\mathbb{Q}}
\newcommand\C{\mathcal{C}}
\let\implies\Rightarrow
\let\impliedby\Leftarrow
\let\iff\Leftrightarrow
\let\epsilon\varepsilon

\newcommand\subs{_{n_k}}
\newcommand\inv{^{-1}}

% theorems
\usepackage{thmtools}
\usepackage[framemethod=TikZ]{mdframed}
\mdfsetup{skipabove=0.5em,skipbelow=0.5em, innertopmargin=5pt, innerbottommargin=5pt}
\pagestyle{empty}
\usepackage{siunitx}
\usepackage{diagbox}
\usepackage{multicol}

\theoremstyle{definition}

\makeatletter

\declaretheoremstyle[headfont=\bfseries, bodyfont=\normalfont, mdframed={ nobreak } ]{thmgreenbox}
\declaretheoremstyle[headfont=\bfseries, bodyfont=\normalfont, mdframed={ nobreak } ]{thmredbox}
\declaretheoremstyle[headfont=\bfseries, bodyfont=\normalfont]{thmbluebox}
\declaretheoremstyle[headfont=\bfseries, bodyfont=\normalfont]{thmblueline}
\declaretheoremstyle[headfont=\bfseries, bodyfont=\normalfont, numbered=no, mdframed={ rightline=false, topline=false, bottomline=false}, qed=\qedsymbol]{thmproofbox}
\declaretheoremstyle[headfont=\bfseries\sffamily, bodyfont=\normalfont, numbered=no, mdframed={ nobreak, rightline=false, topline=false, bottomline=false } ]{thmexplanationbox}

\declaretheorem[numberwithin=section, style=thmgreenbox, name=Definition]{definition}
\declaretheorem[sibling=definition, style=thmredbox, name=Corollary]{corollary}
\declaretheorem[sibling=definition, style=thmredbox, name=Proposition]{prop}
\declaretheorem[sibling=definition, style=thmredbox, name=Theorem]{theorem}
\declaretheorem[sibling=definition, style=thmredbox, name=Lemma]{lemma}

\declaretheorem[numbered=no, style=thmexplanationbox, name=Proof]{explanation}
\declaretheorem[numbered=no, style=thmproofbox, name=Proof]{replacementproof}
\declaretheorem[numbered=no, style=thmproofbox, name=Solution]{replacementsolution}
\declaretheorem[style=thmbluebox,  numbered=no, name=Exercise]{ex}
\declaretheorem[style=thmbluebox,  numbered=no, name=Example]{eg}
\declaretheorem[style=thmblueline, numbered=no, name=Remark]{remark}
\declaretheorem[style=thmblueline, numbered=no, name=Note]{note}

\renewenvironment{proof}[1][\proofname]{\begin{replacementproof}}{\end{replacementproof}}
\renewenvironment{solution}{\begin{replacementsolution}}{\end{replacementsolution}}

\newtheorem*{uovt}{UOVT}
\newtheorem*{notation}{Notation}
\newtheorem*{previouslyseen}{As previously seen}
\newtheorem{problem}{Problem}
\newtheorem*{observe}{Observe}
\newtheorem*{property}{Property}
\newtheorem*{intuition}{Intuition}
\title{Math $451$ \\ Homework $11$}
\author{Michael Williams}
\date{Due: April $18^{\text{th}}$}

\begin{document}

\maketitle
\thispagestyle{empty}

\begin{problem}
\end{problem}

\begin{mybox}


  \textbf{Simple L'Hospital:} Let $f, g : (a, b) \rightarrow \mathbb{R}$
  be continuously differentiable and $c \in (a, b)$.
  Assume that $f(c) = g(c) = 0$ and $g'(c) \neq 0$. Then
  \[
  \lim_{x \to c} \frac{f(x)}{g(x)} = \frac{f'(c)}{g'(c)}.
  \]

  \textbf{Remark:} The more general L'Hospital theorem does not require
  $f, g$ to be defined at $c$, and allows $c = \pm\infty$.
  The proof is more complicated. It can be found in Section 30 of our book.


\end{mybox}

\begin{proof}
  Using the definition of the derivative we have 
  $$\frac{f'(c)}{g'(c)} = \frac{\lim_{x\rightarrow c}
  \frac{f(x)-f(c)}{x-c}}{\lim_{x\rightarrow c}\frac{g(x)-g(c)}{x-c}} =
  lim_{x\rightarrow c}\frac{f(x)-f(c)}{g(x)-g(c)}$$
  Since $f(c)=g(c)=0$ this is equivalent to $$\lim_{x \to c}
   \frac{f(x)}{g(x)}$$
\end{proof}

\begin{rating}
  \Rating{1}
\end{rating}

\begin{problem}
\end{problem}

\begin{mybox}
 
  Is the function $$f(x) = \begin{cases} \sin\left(\frac{1}{x}\right),
  & x \neq 0 \\ 0, & x = 0 \end{cases}$$ defined on the set $[-1, 1]$
  Riemann-integrable? Prove your answer.
  \\
  
  \textbf{Remark:} We have seen that wild oscillations are a problem for
  continuity. Here you'll check if they are a problem for integrability.
  
\end{mybox}


\begin{proof}
  We will show $f(x)$ is Riemann-integrable. Let $\epsilon >0$. 
  First notice that $|f| \leq 1$. Now choose some $c\in\R$ such that 
  $c < \frac{\epsilon}{8}$. Now notice that $f|_{[-1,-c]}$ is continuous
  so there is a partition $$P_1 :=( -1 =a_0^{P_1}, a_1^{P_1},
  ..., a_{|P_1|}^{P_1}= -c)$$ with step functions $h_1$ and $h_2$ adapted
  to $P_1$ such that $h_1 \leq f|_{[-1,-c]} \leq h_2$ where 
  $$\int_{-1}^{-c}h_2(x)\, dx - \int_{-1}^{-c}h_1(x)\, dx < 
  \frac{\epsilon}{4}$$ Similarily, $f|_{[c,1]}$ is continuous so there is
  a partition $$P_2 := (c = a_0^{P_2}, a_1^{P_2}, ..., a_{|P_2|}^{P_2}= 1)$$
  with step functions $h_3$ and $h_4$ adapted to $P_2$ such that 
  $h_3 \leq f|_{[c,1]} \leq h_4$ where $$\int_{c}^{1}h_4(x)\, dx -
  \int_{c}^{1}h_3(x)\, dx < \frac{\epsilon}{4}$$
  
  Now we will define $$P := P_1\cup P_2$$ Now we construct the step 
  functions $h_5$ and $h_6$ adapted to $P$ such that 
  \begin{align*} 
    h_5\bigg|_{(a_i^P, a_{i+1}^P)} = 
    \begin{cases} 
      h_1|_{(a_i^P, a_{i+1}^P)}, & a_i^P < -c \\ 
      \inf f([a_i^P, a_{i+1}^P]), & a_i^P= -c  \\ 
      h_3|_{(a_i^P, a_{i+1}^P)}, & a_i^P > -c 
    \end{cases}
    \\ h_6\bigg|_{(a_i^P, a_{i+1}^P)} = 
    \begin{cases}
      h_2|_{(a_i^P, a_{i+1}^P)}, & a_i^P < -c \\
      \sup f([a_i^P, a_{i+1}^P]), & a_i^P = -c  \\
      h_4|_{(a_i^P, a_{i+1}^P)}, & a_i^P > -c 
    \end{cases}
  \end{align*} 
  By construction we have $h_5 \leq f \leq h_6$. Now we can take 
  \begin{align*}
    \int_{-1}^{1}h_6(x)\, dx &- \int_{-1}^{1}h_5(x)\, dx \\
    = \int_{-1}^{-c}h_2(x)\, dx + \sup f([-c, c])\cdot (c-(-c)) +
    \int_{c}^{1}h_4(x)\, dx -& \left(\int_{-1}^{-c}h_1(x)\, dx +
    \inf f([-c, c])\cdot(c-(-c)) +\int_{c}^{1}h_3(x)\, dx\right) \\
    = \int_{-1}^{-c}h_2(x)\, dx - \int_{-1}^{-c}h_1(x)\, dx 
    + \int_{c}^{1}h_4(x)\, dx &- \int_{c}^{1}h_3(x)\, dx + 2c \cdot
    (\sup f([-c, c]) - \inf f([-c, c])) \\ < \ \frac{\epsilon}{4} +
    \frac{\epsilon}{4} + 2\cdot(\frac{\epsilon}{8}) &\cdot 
    (1 - (-1)) = \epsilon
  \end{align*}
  We can conclude that $f$ is Riemann-integrable.
\end{proof}

\begin{rating}
  \Rating{4}
\end{rating}

\pagebreak

\begin{problem}
\end{problem}


\begin{mybox}
  
  Let $f : [0, 1] \rightarrow \mathbb{R}$ be continuous.
  \begin{enumerate}
    \item[(a)] Assume $f(x) \geq 0$ for all $x$. Show that if
    $\int_{0}^{1} f(x) \, dx = 0$ then $f(x) = 0$ for all $x$.
    \item[(b)] Without assuming $f(x) \geq 0$ for all $x$, show
    that if $\int_{a}^{b} f(x) \, dx = 0$ for all $0 \leq a < b \leq 1$
    then $f(x) = 0$ for all $x$.
  \end{enumerate}

\end{mybox}

\begin{proof}
  \begin{enumerate} 
    \item Let us assume, for the sake of contradiction, that there
    exists some $c\in [0,1]$ such that $f(c)>0$. Consider some
    $0 <\epsilon < f(c)$. First consider the case where $c\in(0,1)$.
    By the continuity of $f$ we can find some $\delta >0$ such
    that $$x\in (c-\delta, c+\delta)\subset (0,1)\implies 
    |(f(x)-f(c))|< \epsilon \implies f(x) >0$$ We thus have
    $$\int_{c-\delta}^{c+\delta}f(x) \, dx > 0$$ But since 
    $f(x)\geq 0$ for all $x$ this implies that $$\int_0^1 f(x) \, 
    dx >0$$ This is a contradiction. 
    
    In the case where $c=0$ we can just find some $\delta >0$ such
    that $$x\in [c, c+\delta)\subset (0,1)\implies |(f(x)-f(c))|
    < \epsilon \implies f(x) >0$$ We can then proceed with same logic
    as above and find a contradiction.

    In the case where $c=1$ we can just find some $\delta >0$ such
    that $$x\in (c-\delta, c]\subset (0,1)\implies |(f(x)-f(c))|<
    \epsilon \implies f(x) >0$$ We can then proceed with same logic
    as above and find a contradiction.

    In every case we have found a contradiction. Thus we can conclude
    that $f(x) = 0$ for all $x$.

    \item  Let us assume, for the sake of contradiction, that there
    exists some $c\in[0, 1]$ such that $f(c) \neq 0$. Without loss
    of generality, suppose that $f(c) > 0$; if $f(c) < 0$, we can
    consider the function $-f$, which would still satisfy the
    integral condition, and proceed analogously.
    
    First consider the case where $c\in (0,1)$.
    Since $f$ is continuous, we can find some $\delta > 0$ such that
    $$x\in (c-\delta, c+\delta)\subset (0,1)\implies |f(x)-f(c)| 
    <\frac{f(c)}{2} \implies f(x) > \frac{f(c)}{2}$$
      
    Now we can take
    \[
    \int_{c - \delta}^{c + \delta} f(x) \, dx > \int_{c - \delta}
    ^{c + \delta} \dfrac{f(c)}{2} \, dx = (2\delta) \cdot 
    \dfrac{f(c)}{2} = \delta f(c) > 0
    \]
    This is a contradiction with our initial assumption.

    Now in the case where $c=0$ we can find some $\delta>0$ such
    that $$x\in [c, c+\delta)\subset (0,1)\implies |f(x)-f(c)|
    <\frac{f(c)}{2} \implies f(x) > \frac{f(c)}{2}$$
    We can then proceed with the same argument as above and find a
    contradiction.

    When $c=1$ we can find some $\delta>0$ such that $$x\in
    (c-\delta, c]\subset (0,1)\implies |f(x)-f(c)| 
    <\frac{f(c)}{2} \implies f(x) > \frac{f(c)}{2}$$
    We can then proceed with the same argument as above and find a
    contradiction.

    In every case we have found a contradiction. Thus, we can
    conclude $f(x) = 0$ for all $x$.
  \end{enumerate}
\end{proof}

\begin{rating}
  \Rating{4}
\end{rating}

\begin{problem}
\end{problem}

\begin{mybox}
  
  Prove that any continuous function $f : [a, b] \rightarrow 
  \mathbb{R}$ is the uniform limit of a sequence of step functions.
  \\

  \textbf{Remark:} This gives a different look on the definition of
  the integral as an approximation by integrals of step functions.
  It turns out that the function itself is approximated by step
  functions in a strong enough way.

\end{mybox}

\begin{proof}
  We will construct $(f_n)$ as follows. For each $f_n$ consider
  $\frac{1}{n}$. Since $f$ is continuous we can find some 
  $\delta_n >0$ such that for all $x,y\in [a,b]$ we have 
  $|x-y|<\delta_n \implies |f(x)-f(y)| <\frac{1}{n}$.

  Now let $$s := \sup\{c\in\N : a+c\delta_n<b\}$$ For every 
  $m\in\N_{\leq s}$ we define $x_m := a+ m\delta_n$ and $x_{s+1}:= b$.
  Now we set $$f_n(x) = \begin{cases}f(x_m), & x_m \leq x < x_{m+1} \\
  f(b), & x=b \end{cases}$$

  We will now show that $f_n \rightarrow f$ uniformly. Let 
  $\epsilon >0$ and by Arch Prop find some $N\in\N$ such that 
  $\epsilon > \frac{1}{N}$. By construction $\forall n>N$ we have, 
  $$|f_n(x) - f(x)| = |f(x-a)-f(x)| \text{ where } a<\delta_n$$ 
  Furthermore, by construction we have $$|f(x-a)-f(x)| < \frac{1}{n} 
  < \frac{1}{N} < \epsilon$$

  We can conclude that $f_n \rightarrow f$ uniformly. Thus, any
  continuous function $f : [a, b] \rightarrow \mathbb{R}$ is the
  uniform limit of a sequence of step functions.
\end{proof}

\begin{rating}
  \Rating{4}
\end{rating}

\pagebreak

\begin{problem}
\end{problem}

\begin{mybox}

  Prove the integral test for convergence of series: Let 
  $f : [1, \infty) \rightarrow \mathbb{R}$ be a non-negative
  monotonically decreasing function. Then $\sum_{n=1}^{\infty} f(n)$
  converges if and only if $\int_{1}^{\infty} f(x) \, dx$ converges.

\end{mybox}

\begin{proof}
  First assume that $\sum_{n=1}^{\infty} f(n)$ converges. Notice that
  since $f$ is non-negative, we have $\sum_{n=1}^{\infty} f(n) \geq 0$.
  Since $f$ is non-negative and monotonically decreasing we know for 
  any $n\in\N$  $$f(n) = f(n)\cdot (n+1 - n) \geq \int_n^{n+1}f(x) \, 
  dx = \int_n^{n+1}f(x) \, dx$$ Therefore we have 
  $$\sum_{n=1}^{\infty} f(n)  \geq   \int_{1}^{\infty} f(x) \, dx$$
  Since $f$ is non-negative, we also know $\int_{1}^{\infty} f(x) 
  \, dx \geq 0$. Combining this with the fact that $\int_{1}^{\infty} 
  f(x) \, dx$ is bounded above by $\sum_{n=1}^{\infty} f(n) \in\R$,
  we can conclude that $\int_{1}^{\infty} f(x) \, dx$ converges.

  Now assume that $\int_{1}^{\infty} f(x) \, dx$ converges. Notice that 
  since $f$ is non-negative, we have $\int_1^{\infty} f(n) \geq 0$. 
  Since $f$ is non-negative and monotonically decreasing we know for 
  any $n\in\N$ $$\int_n^{n+1}f(x) \, dx \geq f(n+1)\cdot (n+1-n)= 
  f(n+1)$$ Therefore we have $$\int_{1}^{\infty} f(x) \, dx \geq 
  \sum_{n=2}^{\infty}f(n) \implies \int_{1}^{\infty} f(x) \, dx 
  + f(1) \geq \sum_{n=1}^{\infty}f(n)$$ Since $f$ is non-negative, 
  we also know $\sum_{n=1}^{\infty}f(n) \geq 0$. Combining this with 
  the fact that $\sum_{n=1}^{\infty}f(n)$ is bounded above by 
  $\int_{1}^{\infty} f(x) \, dx + f(1)\in\R$, we can conclude that 
  $\sum_{n=1}^{\infty}f(n)$ converges.

  We can conclude that $\sum_{n=1}^{\infty} f(n)$ converges if and 
  only if $\int_{1}^{\infty} f(x) \, dx$ converges.
\end{proof}

\begin{rating}
  \Rating{2}
\end{rating}

\end{document}