\usepackage[tmargin=0.5in,rmargin=0.75in,lmargin=0.75in,bmargin=0.5in]{geometry}
\usepackage{hyperref}
\hypersetup{
  colorlinks,
  linkcolor={black},
  citecolor={black},
  urlcolor={blue!80!black}
}
\usepackage{amsmath, amsfonts, mathtools, amsthm, amssymb}
\usepackage{pdfpages}
\usepackage{enumitem}
\usepackage{xcolor}
\usepackage{tcolorbox}
\setlist[enumerate]{label=(\alph*)}

\definecolor{mycolor}{RGB}{90, 177, 200}
\definecolor{lightcolor}{RGB}{250, 254, 255}
\newtcolorbox{mybox}[1][]{%
    colback=lightcolor,
    colframe=mycolor,
}

\usepackage{pifont}

% Define a command for rating stars
\newcommand{\Rating}[1]{%
    \hspace{-1em}%
    \foreach \n in {1,...,#1}{\ding{72}\hspace{0.1em}}%
}

% Define the rating environment
\newenvironment{rating}{%
    \par
    \vspace{\baselineskip}
    \begin{center}
    \begin{tcolorbox}[colback=lightcolor,colframe=mycolor, boxrule=0.5mm, boxsep=3mm, left=2mm, right=2mm, top=0.0mm, bottom=2mm, width=2.3in, height=.4in]
        \textbf{Proof Difficulty:} \hspace{0.5em}%
}{%
    \end{tcolorbox}
    \end{center}
    \vspace{\baselineskip}
    \par
}

\newcommand\N{\mathbb{N}}
\newcommand\R{\mathbb{R}}
\newcommand\Z{\mathbb{Z}}
\renewcommand\O{\emptyset}
\renewcommand{\emptyset}{\varnothing}
\newcommand\Q{\mathbb{Q}}
\newcommand\C{\mathcal{C}}
\let\implies\Rightarrow
\let\impliedby\Leftarrow
\let\iff\Leftrightarrow
\let\epsilon\varepsilon

\newcommand\subs{_{n_k}}
\newcommand\inv{^{-1}}

% theorems
\usepackage{thmtools}
\usepackage[framemethod=TikZ]{mdframed}
\mdfsetup{skipabove=0.5em,skipbelow=0.5em, innertopmargin=5pt, innerbottommargin=5pt}
\pagestyle{empty}
\usepackage{siunitx}
\usepackage{diagbox}
\usepackage{multicol}

\theoremstyle{definition}

\makeatletter

\declaretheoremstyle[headfont=\bfseries, bodyfont=\normalfont, mdframed={ nobreak } ]{thmgreenbox}
\declaretheoremstyle[headfont=\bfseries, bodyfont=\normalfont, mdframed={ nobreak } ]{thmredbox}
\declaretheoremstyle[headfont=\bfseries, bodyfont=\normalfont]{thmbluebox}
\declaretheoremstyle[headfont=\bfseries, bodyfont=\normalfont]{thmblueline}
\declaretheoremstyle[headfont=\bfseries, bodyfont=\normalfont, numbered=no, mdframed={ rightline=false, topline=false, bottomline=false}, qed=\qedsymbol]{thmproofbox}
\declaretheoremstyle[headfont=\bfseries\sffamily, bodyfont=\normalfont, numbered=no, mdframed={ nobreak, rightline=false, topline=false, bottomline=false } ]{thmexplanationbox}

\declaretheorem[numberwithin=section, style=thmgreenbox, name=Definition]{definition}
\declaretheorem[sibling=definition, style=thmredbox, name=Corollary]{corollary}
\declaretheorem[sibling=definition, style=thmredbox, name=Proposition]{prop}
\declaretheorem[sibling=definition, style=thmredbox, name=Theorem]{theorem}
\declaretheorem[sibling=definition, style=thmredbox, name=Lemma]{lemma}

\declaretheorem[numbered=no, style=thmexplanationbox, name=Proof]{explanation}
\declaretheorem[numbered=no, style=thmproofbox, name=Proof]{replacementproof}
\declaretheorem[numbered=no, style=thmproofbox, name=Solution]{replacementsolution}
\declaretheorem[style=thmbluebox,  numbered=no, name=Exercise]{ex}
\declaretheorem[style=thmbluebox,  numbered=no, name=Example]{eg}
\declaretheorem[style=thmblueline, numbered=no, name=Remark]{remark}
\declaretheorem[style=thmblueline, numbered=no, name=Note]{note}

\renewenvironment{proof}[1][\proofname]{\begin{replacementproof}}{\end{replacementproof}}
\renewenvironment{solution}{\begin{replacementsolution}}{\end{replacementsolution}}

\newtheorem*{uovt}{UOVT}
\newtheorem*{notation}{Notation}
\newtheorem*{previouslyseen}{As previously seen}
\newtheorem{problem}{Problem}
\newtheorem*{observe}{Observe}
\newtheorem*{property}{Property}
\newtheorem*{intuition}{Intuition}